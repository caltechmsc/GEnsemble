\documentstyle[epsfig,supertab]{article}
%%%%
% $Id: HEADER.TEX,v 1.7 2005/10/11 21:31:57 vriend Exp $
%%%%%%%%%%%%%%%%%%%%%%%%%%%%%%%%%%%%%%%%%%%%%%%%%%%
% This file contains LaTeX textformatting
% commands. To print it on a PostScript(TM)
% printer, run this file through "latex" and
% the output of "latex" through "dvips". A lot
% of other possibilities exist. Ask your local
% TeXnician for advice. There is some coverage in
% the WHAT IF manual.
%%%%%%%%%%%%%%%%%%%%%%%%%%%%%%%%%%%%%%%%%%%%%%%%%%%
% This is the header of a LaTeX file created by
% WHAT IF. If you are not a TeXnician, please
% do not try to understand the commands here. They
% are needed to create a consistent output, while
% the input is (1) Standard LaTeX (2) moderately
% readable. Portions of this top matter were
% copied from a file marked:
%%%%%%%%%%%%%%%%%%%%%%%%%%%%%%%%%%%%%%%%%%%%%%%%%%%
% Comment.sty   version 3.0, 3 September 1992     %
% selectively in/exclude pieces of text           %
%%%%%%%%%%%%%%%%%%%%%%%%%%%%%%%%%%%%%%%%%%%%%%%%%%%
% Copyright status of this file: The information in
% This file is a product of WHAT IF. You may use the
% it in publications provided that the WHAT IF
% program and its authors are appropriately
% acknowledged. This header is considered to be in
% the public domain.
%
% Implementation: Rob W.W. Hooft, 1993--1996
%%%%%%%%%%%%%%%%%%%%%%%%%%%%%%%%%%%%%%%%%%%%%%%%%%%
\def\makeinnocent#1{\catcode`#1=12 }
\def\csarg#1#2{\expandafter#1\csname#2\endcsname}
\def\TreatAsPart#1{\begingroup
    \def\CurrentPart{#1}%
    \let\do\makeinnocent \dospecials
    \makeinnocent\^^L% and whatever other special cases
    \endlinechar`\^^M \catcode`\^^M=12 \xPart}
{\catcode`\^^M=12 \endlinechar=-1 %
 \gdef\xPart#1^^M{\def\test{#1}
      \csarg\ifx{PlainEnd\CurrentPart Test}\test
          \def\next{\endgroup\AfterPart}%
      \else \csarg\ifx{LolliEnd\CurrentPart Test}\test
          \def\next{\endgroup\AfterPart}%
      \else \csarg\ifx{LaLaEnd\CurrentPart Test}\test
            \edef\next{\endgroup\noexpand\AfterPart
                       \noexpand\end{\CurrentPart}}
      \else \ThisPart{#1}\let\next\xPart
      \fi \fi \fi \next}
}
\def\includepart
 #1{\message{Including the #1s}%
    \expandafter\def\csname#1\endcsname{}%
    \expandafter\def\csname end#1\endcsname{}}
\def\excludepart
 #1{\message{Excluding the #1s}%
    \csarg\def{#1}{\let\AfterPart\relax
           \def\ThisPart####1{}\TreatAsPart{#1}}%
    {\escapechar=-1\relax
     \csarg\xdef{PlainEnd#1Test}{\string\\end#1}%
     \csarg\xdef{LolliEnd#1Test}{\string\\#1Stop}%
     \csarg\xdef{LaLaEnd#1Test}{\string\\end\string\{#1\string\}}%
    }}
\def\showsect#1{
  \thesect\gdef\thesect{}
  \thessect\gdef\thessect{}
  \subsubsection{#1}
}
\def\sect#1{
  \gdef\thesect{\pagebreak[2]\section{#1}}
  \gdef\thessect{\subsection{General}}
}
\def\ssect#1{
  \gdef\thessect{\subsection{#1}}
}
\gdef\thesect{\pagebreak[2]\section{WHAT-IF BUG: NO TITLE!!!!!}}
\gdef\thessect{\subsection{General}}
\def\mark#1{{\bf #1:}}
\def\halfpage#1{\makebox[0.5\textwidth]{#1}}
\def\fullpage#1{\makebox[1\textwidth]{#1}}
\def\hdr#1{{\makebox[1\textwidth]{#1}}\\[-0.3\baselineskip]\nopagebreak}
\def\hdrhdr#1#2{{\halfpage{#1}\halfpage{#2}}\\[-0.3\baselineskip]\nopagebreak}
\def\psplot#1{\begin{center}
\mbox{\epsfig{file=#1,height=6.0cm}}
\end{center}}
\def\psstraightplot#1#2{\begin{center}
\mbox{\epsfig{file=#1,width=6cm}}
\mbox{\epsfig{file=#2,width=6cm}}
\end{center}}
%%%%%%%%%%%%%%%%%%%%%%%%%%%%%%%%%%%%%%%%%%%%%%%%%%%%%%%%%%%%%%%%%%%%%%%%
% The command just above this comment defines the way of displaying
% stereo-plots. In "real life" there are two possible ways of doing
% this: Normal, and Cross-eye. The default when this file is unmodified
% is to use normal stereo. If you are able to cross your eyes and that
% way see stereo plots without mechanical help, exchange the "{#1}{#2}"
% in the command just below here, such that it reads "{#2}{#1}".
% theoretically it is very easy to make so-called A-B-A plots, but an
% A4 page is just too narrow to contain three pictures in a row neatly.
%
% If you don't have the epsfig.sty file: don't panic: it should be in
% the whatif/dbdata directory. Copy it to your TEXINPUTS directory and
% you should be all set. If you don't have Rokiki's dvips program, or
% you don't want to convert to postscript, you could define the \psplot
% and psstereoplot macros to do nothing.
%%%%%%%%%%%%%%%%%%%%%%%%%%%%%%%%%%%%%%%%%%%%%%%%%%%%%%%%%%%%%%%%%%%%%%%%
\def\psstereoplot#1#2{\psstraightplot{#1}{#2}}

\parskip .7em
\parindent 0pt
\raggedbottom
%%%%%%%%%%%%%%%%%%%%%%%%%%%%%%%%%%%%%%%%%%%%%%%%%%%%%%%%%%%%%%%%%%%%%%%%
% This marks the end of the top matter. The customizable variables
% follow. For the three next lines, please use either "includepart" or
% "excludepart". Use at least one "includepart"
%%%%%%%%%%%%%%%%%%%%%%%%%%%%%%%%%%%%%%%%%%%%%%%%%%%%%%%%%%%%%%%%%%%%%%%%

\includepart{error}
\includepart{warning}
\includepart{note}

\begin{document}
\title{Report of protein analysis}
\author{By the {\sc WHAT IF} program}

\date{2016-09-27%
\footnote{This report was created by WHAT IF version 20051228-2357}}
\maketitle
\section{Introduction}
This document contains a report of findings by the {\sc what if} program
during the analysis of one or more proteins. Explanations of residue
numbering, severity (Note, Warning, Error), and Z-scores are given in
the appendices at the end of this document.
The WHAT IF web pages hold much more information about structure validation.
\end{description}
%%%%%%%%%%%%%%%%%%%%%%%%%%%%%%%%%%%%%%%%%%%%%%%%%%%%%%%%%%%%%%%%%%%%%
% End of the preface, start of WHAT IF generated output
%%%%%%%%%%%%%%%%%%%%%%%%%%%%%%%%%%%%%%%%%%%%%%%%%%%%%%%%%%%%%%%%%%%%%

\sect{beta1\_tm45interfaceOPM.addH.pdb}
\ssect{Symmetry related problems}
\begin{error}
\showsect{Error: Cell contains short vector}
The unit cell in the CRYST1 card of the PDB file contains
vectors with lengths smaller than 1.54 {\AA}.

Possible cause: Probably one or more of the values is mistyped, or the
CRYST1 card does not conform to the FORMAT given in the PDB
specification.


\parbox{1\textwidth}{
\hdr{The CRYST1 cell dimensions}

\fullpage{
$\begin{array}{lrlrlr}
A =&   0.000&B =&   0.000&C =&   0.000\\
\alpha =&  90.000&\beta =&  90.000&\gamma =&  90.000\\
\end{array}$}
}% End of ParBox

\end{error}

\begin{note}
\showsect{Note: This is NOT an alpha-carbon-only structure}
At least some amino acids were observed that contained more atoms
than just an $\alpha$-carbon.
\end{note}

\ssect{Administrative problems that can generate validation failures}
\begin{note}
\showsect{Note: No strange inter-chain connections detected}
No covalent bonds have been detected between molecules with
non-identical chain identifiers.
\end{note}

\begin{note}
\showsect{Note: No strange inter-chain connections could be corrected}
Either there were no strange inter-chain conections, or they were too
strange for WHAT IF to correct.
\end{note}

\begin{note}
\showsect{Note: No duplicate atom names}
All atom names seem adequately unique.
\end{note}

\begin{note}
\showsect{Note: No alternate atom problems detected}
Either this structure does not contain alternate atoms, or they are all
correct, or the errors have remained unnoticed.
\end{note}

\begin{note}
\showsect{Note: In all cases the primary aternate atom was used}
WHAT IF saw no need to make any alternate atom corrections.
\end{note}

\begin{note}
\showsect{Note: No overlapping non-alternates detected}
Either this structure does not contain overlapping non-alternate atoms,
or they are all correct, or the errors have remained unnoticed.
\end{note}

\begin{note}
\showsect{Note: No residues detected inside ligands}
Either this structure does not contain ligands with amino acid groups
in it, or their naming is proper (enough).
\end{note}

\begin{note}
\showsect{Note: No attached groups interfere with hydrogen bond calculations}
It seems there are no sugars, lipids, etc., bound (very close) to
atoms that otherwise could form hydrogen bonds.
\end{note}

\begin{note}
\showsect{Note: No C-terminal nitrogen detected}
It is becoming habit to indicate that a residue is not the true C-terminus
by including only the backbone N of the next residue. This has not been
observed in this PDB file.
\end{note}

\begin{note}
\showsect{Note: All residues have a complete backbone.}
No residues have missing backbone atoms.
\end{note}

\begin{note}
\showsect{Note: No probable atoms with zero occupancy detected.}
Either there are no atoms with zero occupancy, or they are not present in
the file, or their positions are sufficiently improbable to warrant a
zero occupancy.
\end{note}

\ssect{Descriptive}
\begin{note}
\showsect{Note: Content of the PDB file as interpreted by WHAT IF}
Content of the PDB file as interpreted by WHAT IF
WHAT IF has read your PDB file, and stored it internally in
what is called 'the soup'. The content of this soup is listed here.
An extensive explanation of all frequently used WHAT IF output formats
can be found at http://swift.cmbi.ru.nl/. Look under output formats.


\begin{center}\begin{supertabular}{rr@{ (}r@{) }r@{ (}r@{) }rlrrr}
\multicolumn{10}{c}{'Molecules'} \\ \hline
     1&    1&   36&  202&  237&A&Protein&        &  &$\beta$1_tm45interfa{\ldots}\\
     2&  203&  280&  279&  356&A&Protein&        &  &$\beta$1_tm45interfa{\ldots}\\
     3&  280&   36&  482&  238&B&Protein&        &  &$\beta$1_tm45interfa{\ldots}\\
     4&  483&  280&  559&  356&B&Protein&        &  &$\beta$1_tm45interfa{\ldots}\\
\end{supertabular}\end{center}
\end{note}

\begin{note}
\showsect{Note: Ramachandran plot}
In this Ramachandran plot X-signs represent glycines, squares represent
prolines and small plus-signs represent the other residues. If too many
plus-signs fall outside the contoured areas then the molecule is poorly
refined (or worse).

In a colour picture, the residues that are part of a helix are
shown in blue, strand residues in red.  "Allowed" regions for
helical residues are drawn in blue, for strand residues in red, and
for all other residues in green.

\parbox{1\textwidth}{
\psplot{eps0001.eps}
}% End of ParBox


\parbox{1\textwidth}{
\hdr{Chain identifier: A}
}% End of ParBox

\end{note}

\begin{note}
\showsect{Note: Ramachandran plot}


\parbox{1\textwidth}{
\psplot{eps0002.eps}
}% End of ParBox


\parbox{1\textwidth}{
\hdr{Chain identifier: B}
}% End of ParBox

\end{note}

\begin{note}
\showsect{Note: Secondary structure}
This is the secondary structure according to DSSP. Only helix (H),
overwound or 3/10-helix (3), strand (S), turn (T) and coil (blank)
are shown. [REF]. All DSSP related information can be found at
http://swift.cmbi.ru.nl/gv/dssp/.
This is not really a structure validation option, but a very scattered
secondary structure (i.e. many strands of only a few residues length,
many Ts inside helices, etc) tends to indicate a poor structure.

\hdr{Secondary structure assignment}

\fullpage{\tt\begin{tabular}{r@{}r@{}r@{}r@{}r@{}r@{}r@{}r@{}r@{}r@{}r@{}r@{}r@{}r@{}r@{}r@{}r@{}r@{}r@{}r@{}r@{}r@{}r@{}r@{}r@{}r@{}r@{}r@{}r@{}r@{}r@{}r@{}r@{}r@{}r@{}r@{}r@{}r@{}r@{}r@{}r@{}r@{}r@{}r@{}r@{}r@{}r@{}r@{}r@{}r@{}r@{}r@{}r@{}r@{}r@{}r@{}r@{}r@{}r@{}r}
\multicolumn{10}{r@{}}{  10}&
\multicolumn{10}{r@{}}{  20}&
\multicolumn{10}{r@{}}{  30}&
\multicolumn{10}{r@{}}{  40}&
\multicolumn{10}{r@{}}{  50}&
\multicolumn{10}{r}{  60}
\\
L&S&Q&Q&W&E&A&G&M&S&L&L&M&A&L&V&V&L&L&I&V&A&G&N&V&L&V&I&A&A&I&G&S&T&Q&R&L&Q&T&L&T&N&L&F&I&T&S&L&A&C&A&D&L&V&V&G&L&L&V&V
\\
 &H&H&H&H&H&T&H&H&H&H&H&H&H&H&H&H&H&H&H&H&H&H&H&H&H&H&H&H&H&H&H&T& &T&T&T& &T&3&3&3&T&H&H&H&H&H&H&H&H&H&H&H&H&T&T&T&H&H
\\
\multicolumn{10}{r@{}}{  70}&
\multicolumn{10}{r@{}}{  80}&
\multicolumn{10}{r@{}}{  90}&
\multicolumn{10}{r@{}}{ 100}&
\multicolumn{10}{r@{}}{ 110}&
\multicolumn{10}{r}{ 120}
\\
P&F&G&A&T&L&V&V&R&G&T&W&L&W&G&S&F&L&C&E&L&W&T&S&L&D&V&L&C&V&T&A&S&I&E&T&L&C&V&I&A&I&D&R&Y&L&A&I&T&S&P&F&R&Y&Q&S&L&M&T&R
\\
H&H&H&H&H&H&H&H&T&T&T& & &T& &H&H&H&H&H&H&H&H&H&H&H&H&H&H&H&H&H&H&H&H&H&H&H&H&H&H&H&H&T&T&T&T&T&T&T& &T&T&H&H&H&H&T& &T
\\
\multicolumn{10}{r@{}}{ 130}&
\multicolumn{10}{r@{}}{ 140}&
\multicolumn{10}{r@{}}{ 150}&
\multicolumn{10}{r@{}}{ 160}&
\multicolumn{10}{r@{}}{ 170}&
\multicolumn{10}{r}{ 180}
\\
A&R&A&K&V&I&I&C&T&V&W&A&I&S&A&L&V&S&F&L&P&I&M&M&H&W&W&R&D&E&D&P&Q&A&L&K&C&Y&Q&D&P&G&C&C&D&F&V&T&N&R&A&Y&A&I&A&S&S&I&I&S
\\
T&H&H&H&H&H&H&H&H&H&H&H&H&H&H&H&H&H&H&H&H&H&H&T&T&T&T& & & &T&H&H&H&H&H&H&H&H&T&T&T& & & & & & & &T&T&H&H&H&H&H&H&H&H&H
\\
\multicolumn{10}{r@{}}{ 190}&
\multicolumn{10}{r@{}}{ 200}&
\\
F&Y&I&P&L&L&I&M&I&F&V&A&L&R&V&Y&R&E&A&K&E&Q
\\
T&H&H&H&H&H&H&H&H&H&H&H&H&H&H&H&H&H&T&T& & 
\\
\multicolumn{ 8}{r@{}}{ 210}&
\multicolumn{10}{r@{}}{ 220}&
\multicolumn{10}{r@{}}{ 230}&
\multicolumn{10}{r@{}}{ 240}&
\multicolumn{10}{r@{}}{ 250}&
\multicolumn{10}{r@{}}{ 260}&
\\
V&M&L&M&R&E&H&K&A&L&K&T&L&G&I&I&M&G&V&F&T&L&C&W&L&P&F&F&L&V&N&I&V&N&V&F&N&R&D&L&V&P&D&W&L&F&V&A&F&N&W&L&G&Y&A&N&S&A&M&N
\\
 & & &H&H&H&H&H&H&H&H&H&H&H&H&H&H&H&H&H&H&H&H&H&H&H&H&H&H&H&H&H&H&H&H&H& &T&T&T&T& &H&H&H&H&H&H&H&H&H&H&H&H&H&H&H&H&H&T
\\
\multicolumn{ 8}{r@{}}{ 270}&
\\
P&I&I&Y&C&R&S&P&D&F&R&K&A&F&K&R&L
\\
3&3&3&3&3&3& &H&H&H&H&H&H&H&H& & 
\\
\multicolumn{11}{r@{}}{ 290}&
\multicolumn{10}{r@{}}{ 300}&
\multicolumn{10}{r@{}}{ 310}&
\multicolumn{10}{r@{}}{ 320}&
\multicolumn{10}{r@{}}{ 330}&
\\
L&S&Q&Q&W&E&A&G&M&S&L&L&M&A&L&V&V&L&L&I&V&A&G&N&V&L&V&I&A&A&I&G&S&T&Q&R&L&Q&T&L&T&N&L&F&I&T&S&L&A&C&A&D&L&V&V&G&L&L&V&V
\\
 &H&H&H&H&H&T&H&H&H&H&H&H&H&H&H&H&H&H&H&H&H&H&H&H&H&H&H&H&H&H&H&T& &T&T&T& &T&3&3&3&T&H&H&H&H&H&H&H&H&H&H&H&H&T&T&T&H&H
\\
\multicolumn{11}{r@{}}{ 350}&
\multicolumn{10}{r@{}}{ 360}&
\multicolumn{10}{r@{}}{ 370}&
\multicolumn{10}{r@{}}{ 380}&
\multicolumn{10}{r@{}}{ 390}&
\\
P&F&G&A&T&L&V&V&R&G&T&W&L&W&G&S&F&L&C&E&L&W&T&S&L&D&V&L&C&V&T&A&S&I&E&T&L&C&V&I&A&I&D&R&Y&L&A&I&T&S&P&F&R&Y&Q&S&L&M&T&R
\\
H&H&H&H&H&H&H&H&T&T&T& & &T& &H&H&H&H&H&H&H&H&H&H&H&H&H&H&H&H&H&H&H&H&H&H&H&H&H&H&H&H&T&T&T&T&T&T&T& &T&T&H&H&H&H&T& &T
\\
\multicolumn{11}{r@{}}{ 410}&
\multicolumn{10}{r@{}}{ 420}&
\multicolumn{10}{r@{}}{ 430}&
\multicolumn{10}{r@{}}{ 440}&
\multicolumn{10}{r@{}}{ 450}&
\\
A&R&A&K&V&I&I&C&T&V&W&A&I&S&A&L&V&S&F&L&P&I&M&M&H&W&W&R&D&E&D&P&Q&A&L&K&C&Y&Q&D&P&G&C&C&D&F&V&T&N&R&A&Y&A&I&A&S&S&I&I&S
\\
T&H&H&H&H&H&H&H&H&H&H&H&H&H&H&H&H&H&H&H&H&H&H&T&T&T&T& & & &T&H&H&H&H&H&H&H&H&T&T&T& & & & & & & &T&T&H&H&H&H&H&H&H&H&H
\\
\multicolumn{11}{r@{}}{ 470}&
\multicolumn{10}{r@{}}{ 480}&
\\
F&Y&I&P&L&L&I&M&I&F&V&A&L&R&V&Y&R&E&A&K&E&Q&I
\\
T&H&H&H&H&H&H&H&H&H&H&H&H&H&H&H&H&H&T&T&T&T& 
\\
\multicolumn{ 8}{r@{}}{ 490}&
\multicolumn{10}{r@{}}{ 500}&
\multicolumn{10}{r@{}}{ 510}&
\multicolumn{10}{r@{}}{ 520}&
\multicolumn{10}{r@{}}{ 530}&
\multicolumn{10}{r@{}}{ 540}&
\\
V&M&L&M&R&E&H&K&A&L&K&T&L&G&I&I&M&G&V&F&T&L&C&W&L&P&F&F&L&V&N&I&V&N&V&F&N&R&D&L&V&P&D&W&L&F&V&A&F&N&W&L&G&Y&A&N&S&A&M&N
\\
 & & &H&H&H&H&H&H&H&H&H&H&H&H&H&H&H&H&H&H&H&H&H&H&H&H&H&H&H&H&H&H&H&H&H& &T&T&T&T& &H&H&H&H&H&H&H&H&H&H&H&H&H&H&H&H&H&T
\\
\multicolumn{ 8}{r@{}}{ 550}&
\\
P&I&I&Y&C&R&S&P&D&F&R&K&A&F&K&R&L
\\
3&3&3&3&3&3& &H&H&H&H&H&H&H&H& & 
\\
\end{tabular}}
\end{note}

\ssect{Problems with the SCALE, CRYST, and symmetry information}
\begin{error}
\showsect{Error: Missing unit cell information}
No SCALE matrix is given in the PDB file.
\end{error}

\ssect{Atom coordinate problems and/or unexpected atoms}
\begin{note}
\showsect{Note: No rounded coordinates detected}
No significant rounding of atom coordinates has been detected.
\end{note}

\begin{note}
\showsect{Note: No missing atoms detected}
All expected atoms are present.
\end{note}

\begin{warning}
\showsect{Warning: C-terminal oxygen atoms missing}
The C-atoms listed in the table below belong to a C-terminal residue
in a protein chain, but the C-terminal oxygen ("O2" or "OXT") that it
should be bound to was not found.


\begin{center}\begin{supertabular}{rl@{ (}r@{}c@{) }l@{}rl}
\multicolumn{6}{c}{Residue} & Atom \\ \hline
 202&GLN & 237& &A&  & C\\
 279&LEU & 356& &A&  & C\\
 482&ILE & 238& &B&  & C\\
 559&LEU & 356& &B&  & C\\
\end{supertabular}\end{center}
\end{warning}

\begin{note}
\showsect{Note: No extra C-terminal groups found}
No C-terminal groups are present for non C-terminal residues.
\end{note}

\begin{note}
\showsect{Note: Weights checked OK}
All atomic occupancy factors ('weights') fall in the 0.0--1.0 range.
\end{note}

\begin{warning}
\showsect{Warning: Normal distribution of occupancy values}

The distribution of the occupancy values in this file seems 'normal'.

Be aware that this evaluation is merely the result of comparing this
file with about 500 well-refined high-resolution files in the PDB. If
this file has much higher or much lower resolution than the PDB files
used in WHAT IF's trainig set, non-normal values might very well be
perfectly fine, or normal values might actually be not so normal{\ldots}
\end{warning}

\begin{note}
\showsect{Note: All occupancies seem to add up to 1.0.}
In principle, the occupancy of all alternates of one atom should add up till
1.0. A valid exception is the missing atom (i.e. an atom not seen in the
electron density) that is allowed to have a 0.0 occupancy.
\end{note}

\begin{warning}
\showsect{Warning: Average B-factor problem}
The average B-factor for all buried protein atoms normally lies between
10--20. Values around 3--5 are expected for X-ray studies performed
at liquid nitrogen temperature.

Because of the extreme value for the average B-factor, no further analysis
of the B-factors is performed.

\parbox{1\textwidth}{
Average B-factor for buried atoms : 54.920
}% End of ParBox

\end{warning}

\begin{note}
\showsect{Note: B-factor plot}
The average atomic B-factor per residue is plotted as function of
the residue number.

\parbox{1\textwidth}{
\psplot{sct0001.eps}
}% End of ParBox


\parbox{1\textwidth}{
\hdr{Chain identifier: A}
}% End of ParBox

\end{note}

\begin{note}
\showsect{Note: B-factor plot}


\parbox{1\textwidth}{
\psplot{sct0002.eps}
}% End of ParBox


\parbox{1\textwidth}{
\hdr{Chain identifier: B}
}% End of ParBox

\end{note}

\ssect{Nomenclature related problems}
\begin{note}
\showsect{Note: Valine nomenclature OK}
No errors were detected in valine nomenclature.
\end{note}

\begin{note}
\showsect{Note: Threonine nomenclature OK}
No errors were detected in threonine nomenclature.
\end{note}

\begin{note}
\showsect{Note: Isoleucine nomenclature OK}
No errors were detected in isoleucine nomenclature.
\end{note}

\begin{note}
\showsect{Note: Leucine nomenclature OK}
No errors were detected in leucine nomenclature.
\end{note}

\begin{warning}
\showsect{Warning: Arginine nomenclature problem}
The arginine residues listed in the table below have their N-H-1
and N-H-2 swapped.


\begin{center}\begin{supertabular}{rl@{ (}r@{}c@{) }l@{}r}
\multicolumn{6}{c}{Residue} \\ \hline
 170&ARG & 205& &A&\\
 268&ARG & 345& &A&\\
 449&ARG & 205& &B&\\
 548&ARG & 345& &B&\\
\end{supertabular}\end{center}
\end{warning}

\begin{warning}
\showsect{Warning: Tyrosine convention problem}
The tyrosine residues listed in the table below have their $\chi$-2
not between -90.0 and 90.0


\begin{center}\begin{supertabular}{rl@{ (}r@{}c@{) }l@{}r}
\multicolumn{6}{c}{Residue} \\ \hline
 114&TYR & 149& &A&\\
 172&TYR & 207& &A&\\
 196&TYR & 231& &A&\\
 393&TYR & 149& &B&\\
 451&TYR & 207& &B&\\
 475&TYR & 231& &B&\\
\end{supertabular}\end{center}
\end{warning}

\begin{warning}
\showsect{Warning: Phenylalanine convention problem}
The phenylalanine residues listed in the table below have their
$\chi$-2 not between -90.0 and 90.0.


\begin{center}\begin{supertabular}{rl@{ (}r@{}c@{) }l@{}r}
\multicolumn{6}{c}{Residue} \\ \hline
  44&PHE &  79& &A&\\
 139&PHE & 174& &A&\\
 181&PHE & 216& &A&\\
 190&PHE & 225& &A&\\
 222&PHE & 299& &A&\\
 238&PHE & 315& &A&\\
 323&PHE &  79& &B&\\
 418&PHE & 174& &B&\\
 460&PHE & 216& &B&\\
 469&PHE & 225& &B&\\
 502&PHE & 299& &B&\\
 518&PHE & 315& &B&\\
\end{supertabular}\end{center}
\end{warning}

\begin{warning}
\showsect{Warning: Aspartic acid convention problem}
The aspartic acid residues listed in the table below have their
$\chi$-2 not between -90.0 and 90.0, or their proton on OD1 instead of
OD2.


\begin{center}\begin{supertabular}{rl@{ (}r@{}c@{) }l@{}r}
\multicolumn{6}{c}{Residue} \\ \hline
 151&ASP & 186& &A&\\
 241&ASP & 318& &A&\\
 430&ASP & 186& &B&\\
 521&ASP & 318& &B&\\
\end{supertabular}\end{center}
\end{warning}

\begin{warning}
\showsect{Warning: Glutamic acid convention problem}
The glutamic acid residues listed in the table below have their
$\chi$-3 outside the -90.0 to 90.0 range, or their proton on OE1 instead
of OE2.


\begin{center}\begin{supertabular}{rl@{ (}r@{}c@{) }l@{}r}
\multicolumn{6}{c}{Residue} \\ \hline
 150&GLU & 185& &A&\\
 201&GLU & 236& &A&\\
 429&GLU & 185& &B&\\
 480&GLU & 236& &B&\\
\end{supertabular}\end{center}
\end{warning}

\begin{note}
\showsect{Note: Heavy atom naming OK}
No errors were detected in the atom names for non-hydrogen atoms.
\end{note}

\ssect{Nomenclature related problems}
\begin{note}
\showsect{Note: Chain names are OK}
All chain names assigned to polymer molecules are unique, and all
residue numbers are strictly increasing within each chain.
\end{note}

\ssect{Geometric checks}
\begin{warning}
\showsect{Warning: Unusual bond lengths}
The bond lengths listed in the table below were found to deviate
more than 4 sigma from standard bond lengths (both standard values
and sigma for amino acid residues have been taken from Engh and
Huber [REF], for DNA they were taken from Parkinson et al [REF]). In
the table below for each unusual bond the bond length and the
number of standard deviations it differs from the normal value is
given.

Atom names starting with "-" belong to the previous residue in the
chain. If the second atom name is "--SS", the disulphide bridge has
a deviating length.


\begin{center}\begin{supertabular}{rl@{ (}r@{}c@{) }l@{}rllrr}
\multicolumn{6}{c}{Residue} & \multicolumn{2}{c}{Atom pair} &
Distance & Z-value \\ \hline
 209&HIS & 286& &A&    & ND1& CE1&  1.376&  4.4\\
\end{supertabular}\end{center}
\end{warning}

\begin{warning}
\showsect{Warning: Low bond length variability}
Bond lengths were found to deviate less than normal from the mean
Engh and Huber [REF] and/or Parkinson et al [REF] standard bond
lengths. The RMS Z-score given below is expected to be around 1.0
for a normally restrained data set. The fact that it is lower than
0.667 in this structure might indicate that too-strong constraints
have been used in the refinement. This can only be a problem
for high resolution X-ray structures.

\parbox{1\textwidth}{
 RMS Z-score for bond lengths: 0.313

 RMS-deviation in bond distances: 0.007
}% End of ParBox

\end{warning}

\begin{note}
\showsect{Note: No bond length directionality}
Comparison of bond distances with Engh and Huber [REF] standard
values for protein residues and Parkinson et al [REF] values for
DNA/RNA does not show significant systematic deviations.
\end{note}

\begin{warning}
\showsect{Warning: Unusual bond angles}
The bond angles listed in the table below were found to deviate
more than 4 sigma from standard bond angles (both standard values
and sigma for protein residues have been taken from Engh and Huber
[REF], for DNA/RNA from Parkinson et al [REF]).  In the table below
for each strange angle the bond angle and the number of standard
deviations it differs from the standard values is given. Please
note that disulphide bridges are neglected. Atoms starting with "-"
belong to the previous residue in the sequence.


\begin{center}\begin{supertabular}{rl@{ (}r@{}c@{) }l@{}rlllrr}
\multicolumn{6}{c}{Residue} & \multicolumn{3}{c}{Atom Triplet} &
Bond Angle & Z-value \\ \hline
 209&HIS & 286& &A&    & CB & CG & ND1&130.817&  6.1\\
 209&HIS & 286& &A&    & CB & CG & CD2&122.817& -4.8\\
 489&HIS & 286& &B&    & CB & CG & ND1&131.228&  6.4\\
 489&HIS & 286& &B&    & CB & CG & CD2&122.410& -5.1\\
\end{supertabular}\end{center}
\end{warning}

\begin{warning}
\showsect{Warning: Low bond angle variability}
Bond angles were found to deviate less than normal from the
standard bond angles (normal values for protein residues were taken
from Engh and Huber [REF], for DNA/RNA from Parkinson et al
[REF]). The RMS Z-score given below is expected to be around 1.0
for a normally restrained data set. More common values are around
1.55. The fact that it is lower than 0.667 in this structure might
indicate that too-strong constraints have been used in the
refinement. This can only be a problem for high resolution X-ray
structures.

\parbox{1\textwidth}{
 RMS Z-score for bond angles: 0.556

 RMS-deviation in bond angles: 1.212
}% End of ParBox

\end{warning}

\begin{note}
\showsect{Note: Chirality OK}
All protein atoms have proper chirality.
\end{note}

\begin{note}
\showsect{Note: Improper dihedral angle distribution OK}
The RMS Z-score for all improper dihedrals in the structure is within
normal ranges.

\parbox{1\textwidth}{
 Improper dihedral RMS Z-score : 0.633
}% End of ParBox

\end{note}

\begin{note}
\showsect{Note: Side chain planarity OK}
All of the side chains of residues that have a planar group are
planar within expected RMS deviations.
\end{note}

\begin{note}
\showsect{Note: Atoms connected to aromatic rings OK}
All of the atoms that are connected to planar aromatic rings in side
chains of amino-acid residues are in the plane within expected RMS
deviations.
\end{note}

\ssect{Torsion-related checks}
\begin{note}
\showsect{Note: PRO puckering amplitude OK}
Puckering amplitudes for all PRO residues are within normal ranges.
\end{note}

\begin{warning}
\showsect{Warning: Unusual PRO puckering phases}
The proline residues listed in the table below have a puckering phase
that is not expected to occur in protein structures. Puckering
parameters were calculated by the method of Cremer and Pople
[REF]. Normal PRO rings approximately show a so-called envelope
conformation with the C-$\gamma$ atom above the plane of the ring
($\phi$=+72 degrees), or a half-chair conformation with C-$\gamma$ below
and C-$\beta$ above the plane of the ring ($\phi$=-90 degrees). If $\phi$
deviates strongly from these values, this is indicative of a very
strange conformation for a PRO residue, and definitely requires a
manual check of the data. Be aware that this is a warning with a low
confidence level.
See: Who checks the checkers? Four validation tools applied to eight
atomic resolution structures. K.Wilson, C.Sander, R.W.W.Hooft, G.Vriend,
et al. J.Mol.Biol. (1998) 276,417-436.



\begin{center}\begin{supertabular}{rl@{ (}r@{}c@{) }l@{}rrl}
\multicolumn{6}{c}{Residue} & Pucker Phase & Conformation \\ \hline
 111&PRO & 146& &A&&  12.6&half-chair N/C-$\delta$ (18 degrees)\\
 390&PRO & 146& &B&&  15.4&half-chair N/C-$\delta$ (18 degrees)\\
 524&PRO & 321& &B&& 101.8&envelop C-$\beta$ (108 degrees)\\
\end{supertabular}\end{center}
\end{warning}

\begin{warning}
\showsect{Warning: Torsion angle evaluation shows unusual residues}
The residues listed in the table below contain bad or abnormal
torsion angles.

These scores give an impression of how ``normal'' the torsion
angles in protein residues are. All torsion angles except $\omega$ are
used for calculating a `normality' score. Average values and
standard deviations were obtained from the residues in the WHAT IF
database. These are used to calculate Z-scores.  A residue with a
Z-score of below -2.0 is poor, and a score of less than -3.0 is
worrying.  For such residues more than one torsion angle is in a
highly unlikely position.


\begin{center}\begin{supertabular}{rl@{ (}r@{}c@{) }l@{}rr}
\multicolumn{6}{c}{Residue} & Z-Score \\ \hline
 178&ILE & 213& &A&&-2.4734\\
 457&ILE & 213& &B&&-2.4268\\
  65&THR & 100& &A&&-2.3240\\
 344&THR & 100& &B&&-2.2778\\
 373&ILE & 129& &B&&-2.1930\\
  94&ILE & 129& &A&&-2.1914\\
 392&ARG & 148& &B&&-2.1144\\
 113&ARG & 148& &A&&-2.0850\\
  25&VAL &  60& &A&&-2.0774\\
 318&THR &  74& &B&&-2.0616\\
  39&THR &  74& &A&&-2.0616\\
 304&VAL &  60& &B&&-2.0476\\
\end{supertabular}\end{center}
\end{warning}

\begin{warning}
\showsect{Warning: Backbone torsion angle evaluation shows unusual conformations}
The residues listed in the table below have abnormal backbone torsion
angles.

Residues with ``forbidden'' $\phi$-$\psi$ combinations are listed, as
well as residues with unusual $\omega$ angles (deviating by more than
3 sigma from the normal value). Please note that it is normal if
about 5 percent of the residues is listed here as having unusual
$\phi$-$\psi$ combinations.


\begin{center}\begin{supertabular}{rl@{ (}r@{}c@{) }l@{}rl}
\multicolumn{6}{c}{Residue} & Description \\ \hline
  64&ALA &  99& &A&&$\omega$ poor\\
  73&LEU & 108& &A&&$\omega$ poor\\
 104&ARG & 139& &A&&$\omega$ poor\\
 131&TRP & 166& &A&&$\omega$ poor\\
 181&PHE & 216& &A&&$\omega$ poor\\
 230&PHE & 307& &A&&$\omega$ poor\\
 343&ALA &  99& &B&&$\omega$ poor\\
 352&LEU & 108& &B&&$\omega$ poor\\
 383&ARG & 139& &B&&$\omega$ poor\\
 410&TRP & 166& &B&&$\omega$ poor\\
 460&PHE & 216& &B&&$\omega$ poor\\
 510&PHE & 307& &B&&$\omega$ poor\\
\end{supertabular}\end{center}
\end{warning}

\begin{error}
\showsect{Error: Ramachandran Z-score very low}
The score expressing how well the backbone conformations of all residues
are corresponding to the known allowed areas in the Ramachandran plot is
very low.

\parbox{1\textwidth}{
 Ramachandran Z-score : -5.408
}% End of ParBox

\end{error}

\begin{note}
\showsect{Note: Omega angle restraint OK}
The $\omega$ angles for trans-peptide bonds in a structure is
expected to give a gaussian distribution with the average around
+178 degrees, and a standard deviation around 5.5. In the current
structure the standard deviation agrees with this expectation.

\parbox{1\textwidth}{
 Standard deviation of $\omega$ values : 5.589
}% End of ParBox

\end{note}

\begin{error}
\showsect{Error: chi-1/chi-2 angle correlation Z-score very low}
The score expressing how well the $\chi$-1/$\chi$-2 angles of all residues
are corresponding to the populated areas in the database is
very low.

\parbox{1\textwidth}{
 $\chi$-1/$\chi$-2 correlation Z-score : -4.423
}% End of ParBox

\end{error}

\begin{warning}
\showsect{Warning: Backbone oxygen evaluation}
The residues listed in the table below have an unusual backbone
oxygen position.

For each of the residues in the structure, a search was performed
to find 5-residue stretches in the WHAT IF database with
superposable C-$\alpha$ coordinates, and some constraints on the
neighboring backbone oxygens.

In the following table the RMS distance between the backbone oxygen
positions of these matching structures in the database and the
position of the backbone oxygen atom in the current residue is
given.  If this number is larger than 1.5 a significant number of
structures in the database show an alternative position for the
backbone oxygen.  If the number is larger than 2.0 most matching
backbone fragments in the database have the peptide plane
flipped. A manual check needs to be performed to assess whether the
experimental data can support that alternative as well. The number
in the last column is the number of database hits (maximum 80) used
in the calculation. It is "normal" that some glycine residues show
up in this list, but they are still worth checking!


\begin{center}\begin{supertabular}{rl@{ (}r@{}c@{) }l@{}rrr}
\multicolumn{6}{c}{Residue} & Distance (\AA) & \# hits \\ \hline
 173&ALA & 208& &A&& 1.83&  80\\
 452&ALA & 208& &B&& 1.82&  80\\
 335&GLY &  91& &B&& 1.58&  23\\
  56&GLY &  91& &A&& 1.55&  28\\
\end{supertabular}\end{center}
\end{warning}

\begin{warning}
\showsect{Warning: Unusual rotamers}
The residues listed in the table below have a rotamer that is not
seen very often in the database of solved protein structures.  This
option determines for every residue the position specific $\chi$-1
rotamer distribution.  Thereafter it verified whether the actual
residue in the molecule has the most preferred rotamer or not. If
the actual rotamer is the preferred one, the score is 1.0. If the
actual rotamer is unique, the score is 0.0. If there are two
preferred rotamers, with a population distribution of 3:2 and your
rotamer sits in the lesser populated rotamer, the score will be
0.667. No value will be given if insufficient hits are found in the
database.

It is not necessarily an error if a few residues have rotamer
values below 0.3, but careful inspection of all residues with these
low values could be worth it.


\begin{center}\begin{supertabular}{rl@{ (}r@{}c@{) }l@{}rlr}
\multicolumn{6}{c}{Residue} & Fraction \\ \hline
 372&SER & 128& &B&&  0.34\\
  93&SER & 128& &A&&  0.35\\
 246&TRP & 323& &A&&  0.38\\
\end{supertabular}\end{center}
\end{warning}

\begin{warning}
\showsect{Warning: Unusual backbone conformations}
For the residues listed in the table below, the backbone formed by
itself and two neighboring residues on either side is in a
conformation that is not seen very often in the database of solved
protein structures.  The number given in the table is the number of
similar backbone conformations in the database with the same amino
acid in the center.

For this check, backbone conformations are compared with database
structures using C-$\alpha$ superpositions with some restraints on the
backbone oxygen positions.

A residue mentioned in the table can be part of a strange loop, or
there might be something wrong with it or its directly surrounding
residues. There are a few of these in every protein, but in any
case it is worth looking at!


\begin{center}\begin{supertabular}{rl@{ (}r@{}c@{) }l@{}rr}
\multicolumn{6}{c}{Residue} & \# hits \\ \hline
 118&MET & 153& &A&& 0\\
 146&TRP & 181& &A&& 0\\
 201&GLU & 236& &A&& 0\\
 202&GLN & 237& &A&& 0\\
 203&VAL & 280& &A&& 0\\
 204&MET & 281& &A&& 0\\
 278&ARG & 355& &A&& 0\\
 279&LEU & 356& &A&& 0\\
 280&LEU &  36& &B&& 0\\
 281&SER &  37& &B&& 0\\
 397&MET & 153& &B&& 0\\
 425&TRP & 181& &B&& 0\\
 481&GLN & 237& &B&& 0\\
 482&ILE & 238& &B&& 0\\
 483&VAL & 280& &B&& 0\\
 484&MET & 281& &B&& 0\\
   9&MET &  44& &A&& 1\\
 163&CYS & 198& &A&& 1\\
 182&TYR & 217& &A&& 1\\
 288&MET &  44& &B&& 1\\
 442&CYS & 198& &B&& 1\\
 461&TYR & 217& &B&& 1\\
  38&GLN &  73& &A&& 2\\
  74&TRP & 109& &A&& 2\\
 150&GLU & 185& &A&& 2\\
 166&PHE & 201& &A&& 2\\
 317&GLN &  73& &B&& 2\\
 353&TRP & 109& &B&& 2\\
 429&GLU & 185& &B&& 2\\
 445&PHE & 201& &B&& 2\\
\end{supertabular}\end{center}
\end{warning}

\begin{note}
\showsect{Note: Backbone conformation Z-score OK}
The backbone conformation analysis gives a score that is normal
for well refined protein structures.

\parbox{1\textwidth}{
 Backbone conformation Z-score : -0.433
}% End of ParBox

\end{note}

\ssect{Bump checks}
\begin{error}
\showsect{Error: Abnormally short interatomic distances}
The pairs of atoms listed in the table below have an unusually
short distance.

The contact distances of all atom pairs have been checked. Two
atoms are said to `bump' if they are closer than the sum of their
Van der Waals radii minus 0.40 {\AA}. For hydrogen bonded pairs
a tolerance of 0.55 {\AA} is used.  The first number in the
table tells you how much shorter that specific contact is than the
acceptable limit. The second distance is the distance between the
centers of the two atoms. Although we believe that two water atoms
at 2.4 A distance are too close, we only report water pairs that are
closer than this rather short distance.

The last text-item on each line represents the status of the atom
pair.  The text `INTRA' means that the bump is between atoms that
are explicitly listed in the PDB file. `INTER' means it is an
inter-symmetry bump. If the final column contains the text 'HB',
the bump criterium was relaxed because there could be a hydrogen
bond. Similarly relaxed criteria are used for 1--3 and 1--4
interactions (listed as 'B2' and 'B3', respectively). If the last
column is 'BF', the sum of the B-factors of the atoms is higher
than 80, which makes the appearance of the bump somewhat less
severe because the atoms probably aren't there anyway.

Bumps between atoms for which the sum of their occupancies is lower
than one are not reported. In any case, each bump is listed in only
one direction.


\begin{center}\begin{supertabular}{rl@{ (}r@{}c@{) }l@{}rl@{ -- }rl@{ (}r@{}c@{) }l@{}rlrrll}
\multicolumn{7}{c}{Atom 1} & \multicolumn{7}{c}{Atom 2} & Bump &
Dist & \multicolumn{2}{c}{Status} \\ \hline
 125&VAL & 160& &A&  & CG1  & 126&ILE & 161& &A&  & N    &  0.337&  2.763&INTRA&\\
 147&TRP & 182& &A&  & C    & 169&ASN & 204& &A&  & ND2  &  0.336&  2.764&INTRA&\\
 404&VAL & 160& &B&  & CG1  & 405&ILE & 161& &B&  & N    &  0.334&  2.766&INTRA&\\
 146&TRP & 181& &A&  & O    & 169&ASN & 204& &A&  & ND2  &  0.315&  2.235&INTRA&HB\\
 178&ILE & 213& &A&  & CG2  & 179&ILE & 214& &A&  & N    &  0.309&  2.791&INTRA&\\
 426&TRP & 182& &B&  & C    & 448&ASN & 204& &B&  & ND2  &  0.305&  2.795&INTRA&\\
 457&ILE & 213& &B&  & CG2  & 458&ILE & 214& &B&  & N    &  0.304&  2.796&INTRA&\\
 481&GLN & 237& &B&  & CG   & 482&ILE & 238& &B&  & N    &  0.290&  2.810&INTRA&BF\\
 113&ARG & 148& &A&  & CB   & 392&ARG & 148& &B&  & NE   &  0.286&  2.814&INTRA&BF\\
 113&ARG & 148& &A&  & NE   & 392&ARG & 148& &B&  & NE   &  0.281&  2.719&INTRA&BF\\
 170&ARG & 205& &A&  & CG   & 171&ALA & 206& &A&  & N    &  0.274&  2.826&INTRA&BF\\
 449&ARG & 205& &B&  & CG   & 450&ALA & 206& &B&  & N    &  0.273&  2.827&INTRA&BF\\
 113&ARG & 148& &A&  & NH1  & 389&SER & 145& &B&  & CB   &  0.268&  2.832&INTRA&BF\\
 442&CYS & 198& &B&  & SG   & 444&ASP & 200& &B&  & CB   &  0.260&  3.140&INTRA&BF\\
 163&CYS & 198& &A&  & SG   & 165&ASP & 200& &A&  & CB   &  0.257&  3.143&INTRA&BF\\
 425&TRP & 181& &B&  & O    & 448&ASN & 204& &B&  & ND2  &  0.250&  2.300&INTRA&HB\\
 110&SER & 145& &A&  & CB   & 392&ARG & 148& &B&  & NH1  &  0.248&  2.852&INTRA&BF\\
 113&ARG & 148& &A&  & NE   & 392&ARG & 148& &B&  & CB   &  0.247&  2.853&INTRA&BF\\
 223&THR & 300& &A&  & O    & 227&LEU & 304& &A&  & CB   &  0.232&  2.568&INTRA&BF\\
 503&THR & 300& &B&  & O    & 507&LEU & 304& &B&  & CB   &  0.222&  2.578&INTRA&BF\\
 113&ARG & 148& &A&  & O    & 114&TYR & 149& &A&  & C    &  0.220&  2.580&INTRA&BF\\
 119&THR & 154& &A&  & C    & 121&ALA & 156& &A&  & N    &  0.217&  2.683&INTRA&BF\\
 392&ARG & 148& &B&  & O    & 393&TYR & 149& &B&  & C    &  0.214&  2.586&INTRA&BF\\
 398&THR & 154& &B&  & C    & 400&ALA & 156& &B&  & N    &  0.210&  2.690&INTRA&BF\\
  64&ALA &  99& &A&  & O    &  67&VAL & 102& &A&  & N    &  0.207&  2.343&INTRA&HB\\
\end{supertabular}\end{center}
And so on for a total of   265 lines.
\end{error}

\ssect{Accessibility related checks}
\begin{warning}
\showsect{Warning: Inside/Outside residue distribution unusual}
The distribution of residue types over the inside and the outside of the
protein is unusual. Normal values for the RMS Z-score below are between
0.84 and 1.16. The fact that it is higher in this structure could be
caused by transmembrane helices, by the fact that it is part of a
multimeric active unit, or by mistraced segments in the density.

\parbox{1\textwidth}{
inside/outside RMS Z-score : 1.207
}% End of ParBox

\end{warning}

\ssect{Accessibility related checks}
\begin{note}
\showsect{Note: Inside/Outside RMS Z-score plot}
The Inside/Outside distribution normality RMS Z-score over a 15
residue window is plotted as function of the residue number. High
areas in the plot (above 1.5) indicate unusual inside/outside
patterns.

\parbox{1\textwidth}{
\psplot{sct0003.eps}
}% End of ParBox


\parbox{1\textwidth}{
\hdr{Chain identifier: A}
}% End of ParBox

\end{note}

\ssect{Accessibility related checks}
\begin{note}
\showsect{Note: Inside/Outside RMS Z-score plot}


\parbox{1\textwidth}{
\psplot{sct0004.eps}
}% End of ParBox


\parbox{1\textwidth}{
\hdr{Chain identifier: B}
}% End of ParBox

\end{note}

\ssect{Directional atomics contact analysis}
\begin{warning}
\showsect{Warning: Abnormal packing environment for some residues}
The residues listed in the table below have an unusual packing
environment.

The packing environment of the residues is compared with the
average packing environment for all residues of the same type in
good PDB files.  A low packing score can indicate one of several
things: Poor packing, misthreading of the sequence through the
density, crystal contacts, contacts with a co-factor, or the
residue is part of the active site. It is not uncommon to see a few
of these, but in any case this requires further inspection of the
residue.


\begin{center}\begin{supertabular}{rl@{ (}r@{}c@{) }l@{}rr}
\multicolumn{6}{c}{Residue} & Qualty value \\ \hline
 558&ARG & 355& &B&& -6.88\\
 278&ARG & 355& &A&& -6.86\\
 484&MET & 281& &B&& -6.13\\
 204&MET & 281& &A&& -6.07\\
 481&GLN & 237& &B&& -5.82\\
 352&LEU & 108& &B&& -5.82\\
  73&LEU & 108& &A&& -5.81\\
 268&ARG & 345& &A&& -5.78\\
 548&ARG & 345& &B&& -5.78\\
 159&GLN & 194& &A&& -5.58\\
 438&GLN & 194& &B&& -5.58\\
 399&ARG & 155& &B&& -5.20\\
 120&ARG & 155& &A&& -5.17\\
\end{supertabular}\end{center}
\end{warning}

\begin{note}
\showsect{Note: No series of residues with bad packing environment}
There are no stretches of three or more residues each having a quality
control score worse than -4.0.
\end{note}

\begin{note}
\showsect{Note: Structural average packing environment OK}
The structural average quality control value is within normal ranges.

\parbox{1\textwidth}{
Average for range    1 - 559 :  -1.004
}% End of ParBox

\end{note}

\begin{note}
\showsect{Note: Quality value plot}
The quality value smoothed over a 10 residue window is plotted as
function of the residue number. Low areas in the plot (below
-2.0) indicate "unusual" packing.

\parbox{1\textwidth}{
\psplot{sct0005.eps}
}% End of ParBox


\parbox{1\textwidth}{
\hdr{Chain identifier: A}
}% End of ParBox

\end{note}

\begin{note}
\showsect{Note: Quality value plot}
The quality value smoothed over a 10 residue window is plotted as
function of the residue number. Low areas in the plot (below
-2.0) indicate "unusual" packing.

\parbox{1\textwidth}{
\psplot{sct0006.eps}
}% End of ParBox


\parbox{1\textwidth}{
\hdr{Chain identifier: B}
}% End of ParBox

\end{note}

\begin{warning}
\showsect{Warning: Low packing Z-score for some residues}
The residues listed in the table below have an unusual packing
environment according to the 2nd generation quality check. The score
listed in the table is a packing normality Z-score: positive means
better than average, negative means worse than average. Only residues
scoring less than -2.50 are listed here. These are the "unusual"
residues in the structure, so it will be interesting to take a
special look at them.


\begin{center}\begin{supertabular}{rl@{ (}r@{}c@{) }l@{}rr}
\multicolumn{6}{c}{Residue} & Z-score \\ \hline
 484&MET & 281& &B&& -2.91\\
 204&MET & 281& &A&& -2.90\\
 203&VAL & 280& &A&& -2.52\\
 483&VAL & 280& &B&& -2.52\\
\end{supertabular}\end{center}
\end{warning}

\begin{warning}
\showsect{Warning: Abnormal packing Z-score for sequential residues}
A stretch of at least four sequential residues with a 2nd
generation packing Z-score below -1.75 was found. This could
indicate that these residues are part of a strange loop or that the
residues in this range are incomplete, but it might also be an
indication of misthreading.

The table below lists the first and last residue in each stretch found,
as well as the average residue Z-score of the series.


\begin{center}\begin{supertabular}{rl@{ (}r@{}c@{) }l@{}r@{ --- }rl@{ (}r@{}c@{) }l@{}rr}
\multicolumn{6}{c}{Start residue} & \multicolumn{6}{c}{End residue} &
Av. Z-score \\ \hline
 202&GLN & 237& &A&    & 205&LEU & 282& &A&    & -1.81\\
\end{supertabular}\end{center}
\end{warning}

\begin{note}
\showsect{Note: Structural average packing Z-score OK}
The structural average for the second generation quality control
value is within normal ranges.

\parbox{1\textwidth}{
 All   contacts    : Average = -0.322 Z-score =  -2.16

 BB-BB contacts    : Average = -0.004 Z-score =  -0.16

 BB-SC contacts    : Average = -0.355 Z-score =  -2.34

 SC-BB contacts    : Average = -0.301 Z-score =  -1.89

 SC-SC contacts    : Average = -0.520 Z-score =  -3.00
}% End of ParBox

\end{note}

\begin{note}
\showsect{Note: Second generation quality Z-score plot}
The second generation quality Z-score smoothed over a 10 residue window
is plotted as function of the residue number. Low areas in the plot (below
-1.3) indicate "unusual" packing.

\parbox{1\textwidth}{
\psplot{sct0007.eps}
}% End of ParBox


\parbox{1\textwidth}{
\hdr{Chain identifier: A}
}% End of ParBox

\end{note}

\begin{note}
\showsect{Note: Second generation quality Z-score plot}


\parbox{1\textwidth}{
\psplot{sct0008.eps}
}% End of ParBox


\parbox{1\textwidth}{
\hdr{Chain identifier: B}
}% End of ParBox

\end{note}

\ssect{Final summary}
\begin{note}
\showsect{Note: Summary report for users of a structure}
This is an overall summary of the quality of the structure as
compared with current reliable structures. This summary is most
useful for biologists seeking a good structure to use for modelling
calculations.

The second part of the table mostly gives an impression of how well
the model conforms to common refinement constraint values. The
first part of the table shows a number of constraint-independent
quality indicators.

\parbox{1\textwidth}{

 Structure Z-scores, positive is better than average:

\begin{tabular}{lrc}
  1st generation packing quality :&  -1.259\\
  2nd generation packing quality :&  -2.156\\
  Ramachandran plot appearance   :&  -5.408& (bad)\\
  $\chi$-1/$\chi$-2 rotamer normality  :&  -4.423& (bad)\\
  Backbone conformation          :&  -0.433\\
\end{tabular}

 RMS Z-scores, should be close to 1.0:

\begin{tabular}{lrc}
  Bond lengths                   :&   0.313& (tight)\\
  Bond angles                    :&   0.556& (tight)\\
  Omega angle restraints         :&   1.016\\
  Side chain planarity           :&   0.210& (tight)\\
  Improper dihedral distribution :&   0.633\\
  Inside/Outside distribution    :&   1.207& (unusual)\\
\end{tabular}
}% End of ParBox

\end{note}


% $Id: TRAILER.TEX,v 1.13 2005/10/11 21:31:57 vriend Exp $
%
% If this file is changed, please make corresponding changes in the
% CheckServer file "~/html/whatifrefs.html"
%
\appendix
\section{Explanation of the output}
This report presents the results of dozens of validation options.
Each reported fact has an assigned severity, one of:
\begin{description}
\item[{\bf error}]: severe errors encountered during the analyses. Items
   marked as errors are considered severe problems requiring immediate
   attention.
\item[{\bf warning}]: Either less severe problems or uncommon structural
   features. These still need special attention.
\item[{\bf note}]: Statistical values, plots, or other verbose results of
   tests and analyses that have been performed. If no error was found, this
   will also be listed as a note.
\end{description}

If alternate conformations are present, only the first is evaluated.
However, WHAT IF will attempt to intelligently select a consistent set
of atoms from among the alternates.

Hydrogen atoms are only included if explicitly requested, and even then
they are not used by all checks.

\subsection{legend}
Some notations need a little explanation:
\begin{description}
\item[Residue] Residues in tables are normally given in 3--5 parts:
\begin{itemize}
\item A number. This is the internal sequence number of the residue used
      by WHAT IF. The first residue in the file gets number 1, etc.
\item The residue name. Normally this is a three letter amino acid name.
\item The sequence number, between brackets. This is the residue number
      as it was given in the input PDB file. It can be followed by an
      insertion code.
\item The chain identifier. A single character. If no chain identifier
      was given in the input file, this will be a blank (invisible).
\item A model number (only for NMR structures).
\end{itemize}
\item[Z-Value] To indicate the normality of a score, the score
   may be expressed as a Z-value or Z-score. This is just the number
   of standard deviations that the score deviates from the expected
   value.  A property of Z-values is that the root-mean-square of a
   group of Z-values (the RMS Z-value) is expected to be 1.0. Z-values
   above 4.0 and below $-4.0$ are very uncommon. If a Z-score is used
   in WHAT IF, the accompanying text will explain how the expected
   value and standard deviation were obtained.
\section{References}
\raggedright

\begin{description}

\item [WHAT IF]
G.Vriend,
{\em WHAT IF: a molecular modelling and drug design program}
J.~Mol. Graph. {\bf 8} 52--56 (1990).

\item [WHAT\_CHECK (verification routines from WHAT IF)]
R.W.W.Hooft, G.Vriend, C.Sander and E.E.Abola,
{\em Errors in protein structures}
Nature {\bf 381}, 272 (1996).

\item [Bond lengths and angles, protein residues]
R.Engh and R.Huber,
{\em Accurate bond and angle parameters for X-ray protein structure refinement}
Acta Crystallogr. {\bf A47}, 392--400 (1991).

\item [Bond lengths and angles, DNA/RNA]
G.Parkinson, J.Voitechovsky, L.Clowney, A.T.Br\"unger and H.Berman,
{\em New parameters for the refinement of nucleic acid-containing structures}
Acta Crystallogr. {\bf D52}, 57--64 (1996).

\item [DSSP]
W.Kabsch and C.Sander,
{\em Dictionary of protein secondary structure: pattern
     recognition of hydrogen bond and geometrical features}
Biopolymers {\bf 22}, 2577--2637 (1983).

\item [Hydrogen bond networks]
R.W.W.Hooft, C.Sander and G.Vriend,
{\em Positioning hydrogen atoms by optimizing hydrogen bond networks in
protein structures}
PROTEINS, {\bf 26}, 363--376 (1996).

\item [Matthews' Coefficient]
B.W.Matthews,
{\em Solvent Content of Protein Crystals}
J.~Mol.~Biol, {\bf 33}, 491--497 (1968).

\item [Protein side chain planarity]
R.W.W. Hooft, C. Sander and G. Vriend,
{\em Verification of protein structures: side-chain planarity}
J.~Appl.~Cryst, {\bf 29}, 714--716 (1996).

\item [Puckering parameters]
D.Cremer and J.A.Pople,
{\em A general definition of ring puckering coordinates}
J.~Am.~Chem.~Soc. {\bf 97}, 1354--1358 (1975).

\item [Quality Control]
G.Vriend and C.Sander,
{\em Quality control of protein models: directional atomic
     contact analysis}
J.~Appl. Cryst. {\bf 26}, 47--60 (1993).

\item [Ramachandran plot]
G.N.Ramachandran, C.Ramakrishnan and V.Sasisekharan.
{\em Stereochemistry of Polypeptide Chain Conformations}
J.~Mol. Biol. {\bf 7} 95--99 (1963).

\item [Symmetry Checks]
R.W.W.Hooft, C.Sander and G.Vriend,
{\em Reconstruction of symmetry related molecules from protein
     data bank {(PDB)} files}
J.~Appl. Cryst. {\bf 27}, 1006--1009 (1994).

\end{description}
\end{document}
