%% LyX 1.3 created this file.  For more info, see http://www.lyx.org/.
%% Do not edit unless you really know what you are doing.
\documentclass[english]{article}
\usepackage[T1]{fontenc}
\usepackage[latin1]{inputenc}
\usepackage{geometry}
\geometry{verbose,letterpaper,tmargin=1in,bmargin=1in,lmargin=1.25in,rmargin=1.25in}
\setlength\parskip{\medskipamount}
\setlength\parindent{0pt}
\IfFileExists{url.sty}{\usepackage{url}}
                      {\newcommand{\url}{\texttt}}

\makeatletter

%%%%%%%%%%%%%%%%%%%%%%%%%%%%%% LyX specific LaTeX commands.
\providecommand{\LyX}{L\kern-.1667em\lower.25em\hbox{Y}\kern-.125emX\@}

%%%%%%%%%%%%%%%%%%%%%%%%%%%%%% Textclass specific LaTeX commands.
 \newenvironment{lyxlist}[1]
   {\begin{list}{}
     {\settowidth{\labelwidth}{#1}
      \setlength{\leftmargin}{\labelwidth}
      \addtolength{\leftmargin}{\labelsep}
      \renewcommand{\makelabel}[1]{##1\hfil}}}
   {\end{list}}

%%%%%%%%%%%%%%%%%%%%%%%%%%%%%% User specified LaTeX commands.
\usepackage{ae,aecompl}
\usepackage{hyperref}
\usepackage{html}

\usepackage{babel}
\makeatother
\begin{document}

\title{IPython\\
{\Large New design notes}}


\author{Fernando P�rez}

\maketitle

\section{Introduction}

This is a draft document with notes and ideas for the IPython rewrite. The section
order and structure of this document roughly reflects in which order things should
be done and what the dependencies are. This document is mainly a draft for developers,
a pdf version is provided with the standard distribution in case regular users
are interested and wish to contribute ideas.

A tentative plan for the future:

\begin{itemize}
\item 0.6.x series: in practice, enough people are using IPython for real work that I
think it warrants a higher number. This series will continue to evolve with bugfixes
and incremental improvements.
\item 0.7.x series: (maybe) If resources allow, there may be a branch for 'unstable'
development, where the architectural rewrite may take place.
\end{itemize}
However, I am starting to doubt it is feasible to keep two separate branches. I
am leaning more towards a \LyX-like approach, where the main branch slowly transforms
and evolves. Having CVS support now makes this a reasonable alternative, as I don't
have to make pre-releases as often. The active branch can remain the mainline of
development, and users interested in the bleeding-edge stuff can always grab the
CVS code.

Ideally, IPython should have a clean class setup that would allow further extensions
for special-purpose systems. I view IPython as a base system that provides a great
interactive environment with full access to the Python language, and which could
be used in many different contexts. The basic hooks are there: the magic extension
syntax and the flexible system of recursive configuration files and profiles. But
with a code as messy as the current one, nobody is going to touch it.


\section{Immediate TODO and bug list}

Things that should be done for the current series, before starting major changes.

\begin{itemize}
\item Fix any bugs reported at the online bug tracker.
\item History bug: I often see that, under certain circumstances, the input history is
incorrect. The problem is that so far, I've failed to find a simple way to reproduce
it consistently, so I can't easily track it down. It seems to me that it happens
when output is generated multiple times for the same input (for i in range(10):
i will do it). But even this isn't reliable... Ultimately the right solution for
this is to cleanly separate the dataflow for input/output history management; right
now that happens all over the place, which makes the code impossible to debug,
and almost guaranteed to be buggy in the first place.
\item \textbf{Redesign the output traps.} They cause problems when users try to execute
code which relies on sys.stdout being the 'true' sys.stdout. They also prevent
scripts which use raw\_input() to work as command-line arguments.\\
The best solution is probably to print the banner first, and then just execute
all the user code straight with no output traps at all. Whatever comes out comes
out. This makes the ipython code actually simpler, and eliminates the problem altogether.\\
These things need to be ripped out, they cause no end of problems. For example,
if user code requires acces to stdin during startup, the process just hangs indefinitely.
For now I've just disabled them, and I'll live with the ugly error messages.
\item The prompt specials dictionary should be turned into a class which does proper
namespace management, since the prompt specials need to be evaluated in a certain
namespace. Currently it's just globals, which need to be managed manually by code
below. 
\item Fix coloring of prompts: the pysh color strings don't have any effect on prompt
numbers, b/c these are controlled by the global scheme. Make the prompts fully
user-definable, colors and all. This is what I said to a user:\\
As far as the green \textbackslash{}\#, this is a minor bug of the coloring code
due to the vagaries of history. While the color strings allow you to control the
coloring of most elements, there are a few which are still controlled by the old
ipython internal coloring code, which only accepts a global 'color scheme' choice.
So basically the input/output numbers are hardwired to the choice in the color
scheme, and there are only 'Linux', 'LightBG' and 'NoColor' schemes to choose from. 
\item Clean up FakeModule issues. Currently, unittesting with embedded ipython breaks
because a FakeModule instance overwrites \_\_main\_\_. Maybe ipython should revert
back to using \_\_main\_\_ directly as the user namespace? Handling a separate
namespace is proving \emph{very} tricky in all corner cases.
\item Make the output cache depth independent of the input one. This way one can have
say only the last 10 results stored and still have a long input history/cache.
\item Fix the fact that importing a shell for embedding screws up the command-line history.
This can be done by not importing the history file when the shell is already inside
ipython.
\item Lay out the class structure so that embedding into a gtk/wx/qt app is trivial,
much like the multithreaded gui shells now provide command-line coexistence with
the gui toolkits. See \url{http://www.livejournal.com/users/glyf/32396.html}
\item Get Holger's completer in, once he adds filename completion.
\end{itemize}
Lower priority stuff:

\begin{itemize}
\item Add @showopt/@setopt (decide name) for viewing/setting all options. The existing
option-setting magics should become aliases for setopt calls.
\item It would be nice to be able to continue with python stuff after an @ command. For
instance \char`\"{}@run something; test\_stuff()\char`\"{} in order to test stuff
even faster. Suggestion by Kasper Souren <Kasper.Souren@ircam.fr>
\item Run a 'first time wizard' which configures a few things for the user, such as color\_info,
editor and the like.
\item Logging: @logstart and -log should start logfiles in \textasciitilde{}.ipython,
but with unique names in case of collisions. This would prevent ipython.log files
all over while also allowing multiple sessions. Also the -log option should take
an optional filename, instead of having a separate -logfile option.\\
In general the logging system needs a serious cleanup. Many functions now in Magic
should be moved to Logger, and the magic @s should be very simple wrappers to the
Logger methods.
\end{itemize}

\section{Lighten the code}

If we decide to base future versions of IPython on Python 2.3, which has the new
Optik module (called optparse), it should be possible to drop DPyGetOpt. We should
also remove the need for Itpl. Another area for trimming is the Gnuplot stuff:
much of that could be merged into the mainline project.

Double check whether we really need FlexCompleter. This was written as an enhanced
rlcompleter, but my patches went in for python 2.2 (or 2.3, can't remember).

With these changes we could shed a fair bit of code from the main trunk.


\section{Unit testing}

All new code should use a testing framework. Python seems to have very good testing
facilities, I just need to learn how to use them. I should also check out QMTest
at \url{http://www.codesourcery.com/qm/qmtest}, it sounds interesting (it's Python-based
too).


\section{Configuration system}

Move away from the current ipythonrc format to using standard python files for
configuration. This will require users to be slightly more careful in their syntax,
but reduces code in IPython, is more in line with Python's normal form (using the
\$PYTHONSTARTUP file) and allows much more flexibility. I also think it's more
'pythonic', in using a single language for everything.

Options can be set up with a function call which takes keywords and updates the
options Struct.

In order to maintain the recursive inclusion system, write an 'include' function
which is basically a wrapper around safe\_execfile(). Also for alias definitions
an alias() function will do. All functionality which we want to have at startup
time for the users can be wrapped in a small module so that config files look like:

\texttt{from IPython.Startup import {*}}~\\
\texttt{...}~\\
\texttt{set\_options(automagic=1,colors='NoColor',...)}~\\
\texttt{...}~\\
\texttt{include('mysetup.py')}~\\
\texttt{...}~\\
\texttt{alias('ls ls -{}-color -l')}~\\
\texttt{... etc.}

Also, put \textbf{all} aliases in here, out of the core code.

The new system should allow for more seamless upgrading, so that:

\begin{itemize}
\item It automatically recognizes when the config files need updating and does the upgrade.
\item It simply adds the new options to the user's config file without overwriting it.
The current system is annoying since users need to manually re-sync their configuration
after every update.
\item It detects obsolete options and informs the user to remove them from his config
file.
\end{itemize}
Here's a copy of Arnd Baecker suggestions on the matter:

1.) upgrade: it might be nice to have an \char`\"{}auto\char`\"{} upgrade procedure:
i.e. imagine that IPython is installed system-wide and gets upgraded, how does
a user know, that an upgrade of the stuff in \textasciitilde{}/.ipython is necessary
? So maybe one has to a keep a version number in \textasciitilde{}/.ipython and
if there is a mismatch with the started ipython, then invoke the upgrade procedure.

2.) upgrade: I find that replacing the old files in \textasciitilde{}/.ipython
(after copying them to .old not optimal (for example, after every update, I have
to change my color settings (and some others) in \textasciitilde{}/.ipython/ipthonrc).
So somehow keeping the old files and merging the new features would be nice. (but
how to distinguish changes from version to version with changes made by the user
?) For, example, I would have to change in GnuplotMagic.py gnuplot\_mouse to 1
after every upgrade ...

This is surely a minor point - also things will change during the \char`\"{}BIG\char`\"{}
rewrite, but maybe this is a point to keep in mind for this ?

3.) upgrade: old, sometimes obsolete files stay in the \textasciitilde{}/.ipython
subdirectory. (hmm, maybe one could move all these into some subdirectory, but
which name for that (via version-number ?) ?)


\subsection{Command line options}

It would be great to design the command-line processing system so that it can be
dynamically modified in some easy way. This would allow systems based on IPython
to include their own command-line processing to either extend or fully replace
IPython's. Probably moving to the new optparse library (also known as optik) will
make this a lot easier.


\section{OS-dependent code}

Options which are OS-dependent (such as colors and aliases) should be loaded via
include files. That is, the general file will have:

\texttt{if os.name == 'posix':}~\\
\texttt{include('ipythonrc-posix.py')}~\\
\texttt{elif os.name == 'nt':}~\\
\texttt{include('ipythonrc-nt.py')...}

In the \texttt{-posix}, \texttt{-nt}, etc. files we'll set all os-specific options.


\section{Merging with other shell systems}

This is listed before the big design issues, as it is something which should be
kept in mind when that design is made.

The following shell systems are out there and I think the whole design of IPython
should try to be modular enough to make it possible to integrate its features into
these. In all cases IPython should exist as a stand-alone, terminal based program.
But it would be great if users of these other shells (some of them which have very
nice features of their own, especially the graphical ones) could keep their environment
but gain IPython's features.

\begin{lyxlist}{00.00.0000}
\item [IDLE]This is the standard, distributed as part of Python. 
\item [pyrepl]\url{http://starship.python.net/crew/mwh/hacks/pyrepl.html}. This is
a text (curses-based) shell-like replacement which doesn't have some of IPython's
features, but has a crucially useful (and hard to implement) one: full multi-line
editing. This turns the interactive interpreter into a true code testing and development
environment. 
\item [PyCrust]\url{http://sourceforge.net/projects/pycrust}. Very nice, wxWindows
based system.
\item [PythonWin]\url{http://starship.python.net/crew/mhammond}. Similar to PyCrust
in some respects, a very good and free Python development environment for Windows
systems.
\end{lyxlist}

\section{Class design}

This is the big one. Currently classes use each other in a very messy way, poking
inside one another for data and methods. ipmaker() adds tons of stuff to the main
\_\_IP instance by hand, and the mix-ins used (Logger, Magic, etc) mean the final
\_\_IP instance has a million things in it. All that needs to be cleanly broken
down with well defined interfaces amongst the different classes, and probably no
mix-ins.

The best approach is probably to have all the sub-systems which are currently mixins
be fully independent classes which talk back only to the main instance (and \textbf{not}
to each other). In the main instance there should be an object whose job is to
handle communication with the sub-systems.

I should probably learn a little UML and diagram this whole thing before I start
coding.


\subsection{Magic}

Now all methods which will become publicly available are called Magic.magic\_name,
the magic\_ should go away. Then, Magic instead of being a mix-in should simply
be an attribute of \_\_IP:

\_\_IP.Magic = Magic()

This will then give all the magic functions as \_\_IP.Magic.name(), which is much
cleaner. This will also force a better separation so that Magic doesn't poke inside
\_\_IP so much. In the constructor, Magic should get whatever information it needs
to know about \_\_IP (even if it means a pointer to \_\_IP itself, but at least
we'll know where it is. Right now since it's a mix-in, there's no way to know which
variables belong to whom).

Build a class MagicFunction so that adding new functions is a matter of:

\texttt{my\_magic = MagicFunction(category = 'System utilities')}~\\
\texttt{my\_magic.\_\_call\_\_ = ...}

Features:

\begin{itemize}
\item The class constructor should automatically register the functions and keep a table
with category sections for easy sorting/viewing.
\item The object interface must allow automatic building of a GUI for them. This requires
registering the options the command takes, the number of arguments, etc, in a formal
way. The advantage of this approach is that it allows not only to add GUIs to the
magics, but also for a much more intelligent building of docstrings, and better
parsing of options and arguments.
\end{itemize}
Also think through better an alias system for magics. Since the magic system is
like a command shell inside ipython, the relation between these aliases and system
aliases should be cleanly thought out.


\subsection{Color schemes}

These should be loaded from some kind of resource file so they are easier to modify
by the user.


\section{Hooks}

IPython should have a modular system where functions can register themselves for
certain tasks. Currently changing functionality requires overriding certain specific
methods, there should be a clean API for this to be done.


\subsection{whos hook}

This was a very nice suggestion from Alexander Schmolck <a.schmolck@gmx.net>:

2. I think it would also be very helpful if there where some sort of hook for ``whos``
that let one customize display formaters depending on the object type.

For example I'd rather have a whos that formats an array like:

\texttt{Variable Type Data/Length}~\\
\texttt{-{}-{}-{}-{}-{}-{}-{}-{}-{}-{}-{}-{}-{}-{}-{}-{}-{}-{}-{}-{}-{}-{}-{}-{}-{}-{}-{}-{}-{}-
}~\\
\texttt{a array size: 4x3 type: 'Float'}

than

\texttt{Variable Type Data/Length }~\\
\texttt{-{}-{}-{}-{}-{}-{}-{}-{}-{}-{}-{}-{}-{}-{}-{}-{}-{}-{}-{}-{}-{}-{}-{}-{}-{}-{}-{}-{}-{}-}~\\
\texttt{a array {[}{[} 0. 1. 2. 3<...> 8. 9. 10. 11.{]}{]}}


\section{Parallel support}

For integration with graphical shells and other systems, it will be best if ipython
is split into a kernel/client model, much like Mathematica works. This simultaneously
opens the door for support of interactive parallel computing. Currenlty \%bg provides
a threads-based proof of concept, and Brian Granger's XGrid project is a much more
realistic such system. The new design should integrates ideas as core elements.
Some notes from Brian on this topic:

1. How should the remote python server/kernel be designed? Multithreaded? Blocking?
Connected/disconnected modes? Load balancing?

2. What APi/protocol should the server/kernel expose to clients?

3. How should the client classes (which the user uses to interact with the cluster)
be designed?

4. What API should the client classes expose?

5. How should the client API be wrapped in a few simple magic functions?

6. How should security be handled?

7. How to work around the issues of the GIL and threads?

I think the most important things to work out are the client API (\#4) the server/kernel
API/protocol (\#2) and the magic function API (\#5). We should let these determine
the design and architecture of the components.

One other thing. What is your impression of twisted? I have been looking at it
and it looks like a \_very\_ powerful set of tools for this type of stuff. I am
wondering if it might make sense to think about using twisted for this project.


\section{Manuals}

The documentation should be generated from docstrings for the command line args
and all the magic commands. Look into one of the simple text markup systems to
see if we can get latex (for re\LyX{}ing later) out of this. Part of the build
command would then be to make an update of the docs based on this, thus giving
more complete manual (and guaranteed to be in sync with the code docstrings).

{[}PARTLY DONE{]} At least now all magics are auto-documented, works farily well.
Limited Latex formatting yet.


\subsection{Integration with pydoc-help}

It should be possible to have access to the manual via the pydoc help system somehow.
This might require subclassing the pydoc help, or figuring out how to add the IPython
docs in the right form so that help() finds them.

Some comments from Arnd and my reply on this topic:

> ((Generally I would like to have the nice documentation > more easily accessable
from within ipython ... > Many people just don't read documentation, even if it
is > as good as the one of IPython ))

That's an excellent point. I've added a note to this effect in new\_design. Basically
I'd like help() to naturally access the IPython docs. Since they are already there
in html for the user, it's probably a matter of playing a bit with pydoc to tell
it where to find them. It would definitely make for a much cleaner system. Right
now the information on IPython is:

-ipython --help at the command line: info on command line switches 

-? at the ipython prompt: overview of IPython 

-magic at the ipython prompt: overview of the magic system 

-external docs (html/pdf)

All that should be better integrated seamlessly in the help() system, so that you
can simply say:

help ipython -> full documentation access 

help magic -> magic overview 

help profile -> help on current profile 

help -> normal python help access.


\section{Graphical object browsers}

I'd like a system for graphically browsing through objects. \texttt{@obrowse} should
open a widged with all the things which \texttt{@who} lists, but cliking on each
object would open a dedicated object viewer (also accessible as \texttt{@oview
<object>}). This object viewer could show a summary of what \texttt{<object>?}
currently shows, but also colorize source code and show it via an html browser,
show all attributes and methods of a given object (themselves openable in their
own viewers, since in Python everything is an object), links to the parent classes,
etc.

The object viewer widget should be extensible, so that one can add methods to view
certain types of objects in a special way (for example, plotting Numeric arrays
via grace or gnuplot). This would be very useful when using IPython as part of
an interactive complex system for working with certain types of data.

I should look at what PyCrust has to offer along these lines, at least as a starting
point.


\section{Miscellaneous small things}

\begin{itemize}
\item Collect whatever variables matter from the environment in some globals for \_\_IP,
so we're not testing for them constantly (like \$HOME, \$TERM, etc.)
\end{itemize}

\section{Session restoring}

I've convinced myself that session restore by log replay is too fragile and tricky
to ever work reliably. Plus it can be dog slow. I'd rather have a way of saving/restoring
the {*}current{*} memory state of IPython. I tried with pickle but failed (can't
pickle modules). This seems the right way to do it to me, but it will have to wait
until someone tells me of a robust way of dumping/reloading {*}all{*} of the user
namespace in a file.

Probably the best approach will be to pickle as much as possible and record what
can not be pickled for manual reload (such as modules). This is not trivial to
get to work reliably, so it's best left for after the code restructuring.

The following issues exist (old notes, see above paragraph for my current take
on the issue):

\begin{itemize}
\item magic lines aren't properly re-executed when a log file is reloaded (and some of
them, like clear or run, may change the environment). So session restore isn't
100\% perfect.
\item auto-quote/parens lines aren't replayed either. All this could be done, but it
needs some work. Basically it requires re-running the log through IPython itself,
not through python.
\item \_p variables aren't restored with a session. Fix: same as above.
\end{itemize}

\section{Tips system}

It would be nice to have a tip() function which gives tips to users in some situations,
but keeps track of already-given tips so they aren't given every time. This could
be done by pickling a dict of given tips to IPYTHONDIR.


\section{TAB completer}

Some suggestions from Arnd Baecker:

a) For file related commands (ls, cat, ...) it would be nice to be able to TAB
complete the files in the current directory. (once you started typing something
which is uniquely a file, this leads to this effect, apart from going through the
list of possible completions ...). (I know that this point is in your documentation.)

More general, this might lead to something like command specific completion ? 

Here's John Hunter's suggestion:

The {*}right way to do it{*} would be to make intelligent or customizable choices
about which namespace to add to the completion list depending on the string match
up to the prompt, eg programmed completions. In the simplest implementation, one
would only complete on files and directories if the line preceding the tab press
matched 'cd ' or 'run ' (eg you don't want callable showing up in 'cd ca<TAB>')

In a more advanced scenario, you might imaging that functions supplied the TAB
namespace, and the user could configure a dictionary that mapped regular expressions
to namespace providing functions (with sensible defaults). Something like

completed = \{\\
'\textasciicircum{}cd\textbackslash{}s+(.{*})' : complete\_files\_and\_dirs,\\
'\textasciicircum{}run\textbackslash{}s+(.{*})' : complete\_files\_and\_dirs,\\
'\textasciicircum{}run\textbackslash{}s+(-.{*})' : complete\_run\_options,\\
\}

I don't know if this is feasible, but I really like programmed completions, which
I use extensively in tcsh. My feeling is that something like this is eminently
doable in ipython.

/JDH

For something like this to work cleanly, the magic command system needs also a
clean options framework, so all valid options for a given magic can be extracted
programatically.


\section{Debugger}

Current system uses a minimally tweaked pdb. Fine-tune it a bit, to provide at
least:

\begin{itemize}
\item Tab-completion in each stack frame. See email to Chris Hart for details.
\item Object information via ? at least. Break up magic\_oinfo a bit so that pdb can
call it without loading all of IPython. If possible, also have the other magics
for object study: doc, source, pdef and pfile.
\item Shell access via !
\item Syntax highlighting in listings. Use py2html code, implement color schemes.
\end{itemize}

\section{A Python-based system shell - pysh?}

Note: as of IPython 0.6.1, most of this functionality has actually been implemented.

This section is meant as a working draft for discussions on the possibility of
having a python-based system shell. It is the result of my own thinking about these
issues as much of discussions on the ipython lists. I apologize in advance for
not giving individual credit to the various contributions, but right now I don't
have the time to track down each message from the archives. So please consider
this as the result of a collective effort by the ipython user community.

While IPyhton is (and will remain) a python shell first, it does offer a fair amount
of system access functionality:

- ! and !! for direct system access,

- magic commands which wrap various system commands,

- @sc and @sx, for shell output capture into python variables,

- @alias, for aliasing system commands.

This has prompted many users, over time, to ask for a way of extending ipython
to the point where it could be used as a full-time replacement over typical user
shells like bash, csh or tcsh. While my interest in ipython is such that I'll concentrate
my personal efforts on other fronts (debugging, architecture, improvements for
scientific use, gui access), I will be happy to do anything which could make such
a development possible. It would be the responsibility of someone else to maintain
the code, but I would do all necessary architectural changes to ipython for such
an extension to be feasible.

I'll try to outline here what I see as the key issues which need to be taken into
account. This document should be considered an evolving draft. Feel free to submit
comments/improvements, even in the form of patches.

In what follows, I'll represent the hypothetical python-based shell ('pysh' for
now) prompt with '>\,{}>'.


\subsection{Basic design principles}

I think the basic design guideline should be the following: a hypothetical python
system shell should behave, as much as possible, like a normal shell that users
are familiar with (bash, tcsh, etc). This means:

1. System commands can be issued directly at the prompt with no special syntax:

>\,{}> ls

>\,{}> xemacs

should just work like a user expects.

2. The facilities of the python language should always be available, like they
are in ipython:

>\,{}> 3+4 \\
7

3. It should be possible to easily capture shell output into a variable. bash and
friends use backquotes, I think using a command (@sc) like ipython currently does
is an acceptable compromise.

4. It should also be possible to expand python variables/commands in the middle
of system commands. I thihk this will make it necessary to use \$var for name expansions:

>\,{}> var='hello' \# var is a Python variable \\
>\,{}> print var hello \# This is the result of a Python print command
\\
>\,{}> echo \$var hello \# This calls the echo command, expanding
'var'.

5. The above capabilities should remain possible for multi-line commands. One of
the most annoying things I find about tcsh, is that I never quite remember the
syntactic details of looping. I often want to do something at the shell which involves
a simple loop, but I can never remember how to do it in tcsh. This often means
I just write a quick throwaway python script to do it (Perl is great for this kind
of quick things, but I've forgotten most its syntax as well).

It should be possible to write code like:

>\,{}> for ext in {[}'.jpg','.gif'{]}: \\
.. ls file\$ext

And have it work as 'ls file.jpg;ls file.gif'.


\subsection{Smaller details }

If the above are considered as valid guiding principles for how such a python system
shell should behave, then some smaller considerations and comments to keep in mind
are listed below.

- it's ok for shell builtins (in this case this includes the python language) to
override system commands on the path. See tcsh's 'time' vs '/usr/bin/time'. This
settles the 'print' issue and related.

- pysh should take

foo args

as a command if (foo args is NOT valid python) and (foo is in \$PATH).

If the user types

>\,{}> ./foo args

it should be considered a system command always.

- \_, \_\_ and \_\_\_ should automatically remember the previous 3 outputs captured
from stdout. In parallel, there should be \_e, \_\_e and \_\_\_e for stderr. Whether
capture is done as a single string or in list mode should be a user preference.
If users have numbered prompts, ipython's full In/Out cache system should be available.

But regardless of how variables are captured, the printout should be like that
of a plain shell (without quotes or braces to indicate strings/lists). The everyday
'feel' of pysh should be more that of bash/tcsh than that of ipython.

- filename completion first. Tab completion could work like in ipython, but with
the order of priorities reversed: first files, then python names.

- configuration via standard python files. Instead of 'setenv' you'd simply write
into the os.environ{[}{]} dictionary. This assumes that IPython itself has been
fixed to be configured via normal python files, instead of the current clunky ipythonrc
format.

- IPython can already configure the prompt in fairly generic ways. It should be
able to generate almost any kind of prompt which bash/tcsh can (within reason).

- Keep the Magics system. They provide a lightweight syntax for configuring and
modifying the state of the user's session itself. Plus, they are an extensible
system so why not give the users one more tool which is fairly flexible by nature?
Finally, having the @magic optional syntax allows a user to always be able to access
the shell's control system, regardless of name collisions with defined variables
or system commands.

But we need to move all magic functionality into a protected namespace, instead
of the current messy name-mangling tricks (which don't scale well). 


\section{Future improvements}

\begin{itemize}
\item When from <mod> import {*} is used, first check the existing namespace and at least
issue a warning on screen if names are overwritten.
\item Auto indent? Done, for users with readline support.
\end{itemize}

\subsection{Better completion a la zsh}

This was suggested by Arnd:

> >~~~ More general, this might lead to something like

> >~~~ command specific completion ?

>

> I'm not sure what you mean here.

~

Sorry, that was not understandable, indeed ...

I thought of something like

~- cd and then use TAB to go through the list of directories

~- ls and then TAB to consider all files and directories

~- cat and TAB: only files (no directories ...)

~

For zsh things like this are established by defining in .zshrc

~

compctl -g '{*}.dvi' xdvi

compctl -g '{*}.dvi' dvips

compctl -g '{*}.tex' latex

compctl -g '{*}.tex' tex

...


\section{Outline of steps}

Here's a rough outline of the order in which to start implementing the various
parts of the redesign. The first 'test of success' should be a clean pychecker
run (not the mess we get right now).

\begin{itemize}
\item Make Logger and Magic not be mixins but attributes of the main class. 

\begin{itemize}
\item Magic should have a pointer back to the main instance (even if this creates a recursive
structure) so it can control it with minimal message-passing machinery. 
\item Logger can be a standalone object, simply with a nice, clean interface.
\item Logger currently handles part of the prompt caching, but other parts of that are
in the prompts class itself. Clean up.
\end{itemize}
\item Change to python-based config system.
\item Move make\_IPython() into the main shell class, as part of the constructor. Do
this \emph{after} the config system has been changed, debugging will be a lot easier
then.
\item Merge the embeddable class and the normal one into one. After all, the standard
ipython script \emph{is} a python program with IPython embedded in it. There's
no need for two separate classes (\emph{maybe} keep the old one around for the
sake of backwards compatibility).
\end{itemize}

\section{Ville Vainio's suggestions}

Some notes sent in by Ville Vainio \texttt{<vivainio@kolumbus.fi>} on Tue, 29 Jun
2004. Keep here for reference, some of it replicates things already said above.

Current ipython seems to \char`\"{}special case\char`\"{} lots of stuff - aliases,
magics etc. It would seem to yield itself to a simpler and more extensible architecture,
consisting of multple dictionaries, where just the order of search is determined
by profile/prefix. All the functionality would just be \char`\"{}pushed\char`\"{}
to ipython core, i.e. the objects that represent the functionality are instantiated
on \char`\"{}plugins\char`\"{} and they are registered with ipython core. i.e.

def magic\_f(options, args): pass

m = MyMagic(magic\_f) m.arghandler = stockhandlers.OptParseArgHandler m.options
= .... \# optparse options, for easy passing to magic\_f and help display

\# note that arghandler takes a peek at the instance, sees options, and proceeds
\# accordingly. Various arg handlers can ask for arbitrary options. \# some handler
might optionally glob the filenames, search data folders for filenames etc.

ipythonregistry.register(category = \char`\"{}magic\char`\"{}, name = \char`\"{}mymagic\char`\"{},
obj = m)

I bet most of the current functionality could easily be added to such a registry
by just instantiating e.g. \char`\"{}Magic\char`\"{} class and registering all
the functions with some sensible default args. Supporting legacy stuff in general
would be easy - just implement new handlers (arg and otherwise) for new stuff,
and have the old handlers around forever / as long as is deemed appropriate. The
'python' namespace (locals() + globals()) should be special, of course.

It should be easy to have arbitrary number of \char`\"{}categories\char`\"{} (like
'magic', 'shellcommand','projectspecific\_myproject', 'projectspecific\_otherproject').
It would only influence the order in which the completions are suggested, and in
case of name collision which one is selected. Also, I think all completions should
be shown, even the ones in \char`\"{}later\char`\"{} categories in the case of
a match in an \char`\"{}earlier\char`\"{} category.

The \char`\"{}functionality object\char`\"{} might also have a callable object
'expandarg', and ipython would run it (with the arg index) when tab completion
is attempted after typing the function name. It would return the possible completions
for that particular command... or None to \char`\"{}revert to default file completions\char`\"{}.
Such functionality could be useful in making ipython an \char`\"{}operating console\char`\"{}
of a sort. I'm talking about:

>\,{}> lscat reactor \# list commands in category - reactor is \char`\"{}project
specific\char`\"{} category

r\_operate

>\,{}> r\_operate <tab> start shutdown notify\_meltdown evacuate

>\,{}> r\_operate shutdown <tab>

1 2 5 6 \# note that 3 and 4 are already shut down

>\,{}> r\_operate shutdown 2

Shutting down.. ok.

>\,{}> r\_operate start <tab>

2 3 4 \# 2 was shut down, can be started now

>\,{}> r\_operate start 2

Starting.... ok.

I'm talking about having a super-configurable man-machine language here! Like cmd.Cmd
on steroids, as a free addition to ipython!
\end{document}
